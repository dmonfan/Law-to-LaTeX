\documentclass{article}
\usepackage{enumerate}
\usepackage[pdftex,
pdftitle={Law to LaTeX: AI Scholarship for Service Act},
pdfauthor={David Fan},
pdfsubject={Example of bill section text in LaTeX}
pdfkeywords={LaTeX, law, AI, scholarship},
pdfpagemode=UseOutlines,
bookmarks,bookmarksopen,
bookmarksnumbered, % Number sections like "(a) Definitions" in Bookmarks
pdfstartview=FitH
]
{hyperref} 
\usepackage{bookmark}
\usepackage{tocloft}
\usepackage[small]{titlesec} % Change size of section headings

\renewcommand{\thesection}{(\alph{section})} % Change section headings to (a)
\renewcommand{\labelenumii}{(\Alph{enumii})} % Change labels in enum environment to (A)

\setlength{\cftsecnumwidth}{2em} % From tocloft, change the width of section numbers in toc
\setlength{\cftbeforesecskip}{.75em} % Change line spacing of toc so more sections can fit
\title{Law to \LaTeX: AI~Scholarship-for-Service~Act.}

\author{David Fan}

%ref / label
%Section (d) / sec:program
%Section (i) / sec:support
%Section (i)(2) / enum:terms
%Section (l) / sec:repay
%Section (k) / sec:amount
%Appendix I / appendix:ai
%Section (i)(2)(D) / enum:postaward

\begin{document}

\maketitle

\abstract{This is an experiment in converting bill text to \LaTeX\ by hand \& python script. The text is section 2208 of H.R. 4521 EAS in the 117\textsuperscript{th}~Congress, the \emph{AI~Scholarship-for-Service~Act.}}

\bookmark[level=1,page=1,view=FitH \calc{\paperheight-\topmargin-1in-\headheight-\headsep}]{Title} % Add Title Bookmark
\pdfbookmark[1]{\contentsname}{toc} % Add Contents Bookmark

\tableofcontents

    \section{Definitions.}In this section:
\begin{enumerate}
            \item {\bf Artificial intelligence.}---The term ``artificial 
        intelligence'' or ``AI'' has the meaning given the term 
        ``artificial intelligence'' in \hyperref[appendix:ai]{section~238(g)} of the John S. 
        McCain National Defense Authorization Act for Fiscal Year 2019
        (10 U.S.C. 2358 note).
            \item {\bf Executive agency.}---The term ``executive agency'' has 
        the meaning given the term ``Executive agency'' in section 105 
        of title 5, United States Code.\footnote{``Executive agency" means an Executive department, a Government corporation, and an independent establishment.}
            \item {\bf Registered internship.}---The term ``registered 
        internship'' means a Federal Registered Internship Program 
        coordinated through the Department of Labor.
\end{enumerate}
    \section{In General.}The Director, in coordination with the Director of 
the Office of Personnel Management, the Director of the National 
Institute of Standards and Technology, and the heads of other agencies 
with appropriate scientific knowledge, shall establish a Federal 
artificial intelligence scholarship-for-service program (referred to in 
this section as the Federal~AI~Scholarship-for-Service~Program) to 
recruit and train artificial intelligence professionals to lead and 
support the application of artificial intelligence to the missions of 
Federal, State, local, and Tribal governments.
    \section{Qualified Institution of Higher Education.}The Director, in 
coordination with the heads of other agencies with appropriate 
scientific knowledge, shall establish criteria to designate qualified 
institutions of higher education that shall be eligible to participate 
in the Federal AI Scholarship-for-Service program. Such criteria shall 
include---
\begin{enumerate}
            \item measures of the institution's demonstrated excellence 
        in the education of students in the field of artificial 
        intelligence; and
            \item measures of the institution's ability to attract and 
        retain a diverse and non-traditional student population in the 
        fields of science, technology, engineering, and mathematics, 
        which may include the ability to attract women, minorities, and 
        individuals with disabilities.
\end{enumerate}
    \section{Program Description and Components.}\label{sec:program}The Federal AI 
Scholarship-for-Service Program shall---
\begin{enumerate}
            \item provide scholarships through qualified institutions of 
        higher education to students who are enrolled in programs of 
        study at institutions of higher education leading to degrees or 
        concentrations in or related to the artificial intelligence 
        field;
            \item provide the scholarship recipients with summer 
        internship opportunities, registered internships, or other 
        meaningful temporary appointments in the Federal workforce 
        focusing on AI projects or research;
            \item prioritize the employment placement of scholarship 
        recipients in executive agencies;
            \item identify opportunities to promote multi-disciplinary 
        programs of study that integrate basic or advanced AI training 
        with other fields of study, including those that address the 
        social, economic, legal, and ethical implications of human 
        interaction with AI systems; and
            \item support capacity-building education research programs 
        that will enable postsecondary educational institutions to 
        expand their ability to train the next-generation AI workforce, 
        including AI researchers and practitioners.
\end{enumerate}
    \section{Scholarship Amounts.}Each scholarship under \hyperref[sec:program]{subsection (d)} 
shall be in an amount that covers the student's tuition and fees at the 
institution for not more than 3 years and provides the student with an 
additional stipend.
    \section{Post-award Employment Obligations.}Each scholarship recipient, 
as a condition of receiving a scholarship under the program, shall 
enter into an agreement under which the recipient agrees to work for a 
period equal to the length of the scholarship, following receipt of the 
student's degree, in the AI mission of---
\begin{enumerate}
            \item an executive agency;
            \item Congress, including any agency, entity, office, or 
        commission established in the legislative branch;
            \item an interstate agency;
            \item a State, local, or Tribal government, which may include 
        instruction in AI-related skill sets in a public school system; 
        or
            \item a State, local, or Tribal government-affiliated 
        nonprofit entity that is considered to be critical 
        infrastructure (as defined in section~1016(e) of the USA 
        Patriot Act (42 U.S.C.~5195c(e))).\footnote{The term ``critical infrastructure" means systems and assets, whether physical or virtual, so vital to the United States that the incapacity or destruction of such systems and assets would have a debilitating impact on security, national economic security, national public health or safety, or any combination of those matters.}
\end{enumerate}
    \section{Hiring Authority.}
\begin{enumerate}
            \item {\bf Appointment in excepted service.}---Notwithstanding any 
        provision of chapter 33 of title 5, United States Code, 
        governing appointments in the competitive service, an executive 
        agency may appoint an individual who has completed the eligible 
        degree program for which a scholarship was awarded to a 
        position in the excepted service in the executive agency.
            \item {\bf Noncompetitive conversion.}---Except as provided in 
        paragraph~(4), upon fulfillment of the service term, an 
        employee appointed under paragraph~(1) may be converted 
        noncompetitively to term, career-conditional, or career 
        appointment.\footnote{{\em 5 USC 3304a: Competitive service; career appointment after 3 years' temporary service}
        \begin{enumerate}[(a)]
        \item An individual serving in a position in the competitive service under an indefinite appointment or a temporary appointment pending establishment of a register (other than an individual serving under an overseas limited appointment, or in a position classified above GS–15 pursuant to section~5108) acquires competitive status and is entitled to have his appointment converted to a career appointment, without condition, when---
        \begin{enumerate}[(1)]
        \item he completes, without break in service of more than 30 days, a total of at least 3~years of service in such a position;
        \item he passes a suitable noncompetitive examination;
        \item the appointing authority (A) recommends to the Office of Personnel Management that the appointment of the individual be converted to a career appointment and (B) certifies to the Office that the work performance of the individual for the past 12~months has been satisfactory; and
        \item he meets Office qualification requirements for the position and is otherwise eligible for career appointment.
        \end{enumerate}
        \end{enumerate}
        }
            \item {\bf Timing of conversion.}---An executive agency may 
        noncompetitively convert a term employee appointed under 
        paragraph~(2) to a career-conditional or career appointment 
        before the term appointment expires.
            \item {\bf Authority to decline conversion.}---An executive agency 
        may decline to make the noncompetitive conversion or 
        appointment under paragraph~(2) for cause.
\end{enumerate}
    \section{Eligibility.}To be eligible to receive a scholarship under 
this section, an individual shall---
\begin{enumerate}
            \item be a citizen or lawful permanent resident of the United 
        States;
            \item demonstrate a commitment to a career in advancing the 
        field of AI;
            \item be---
\begin{enumerate}
                    \item a full-time student in an eligible degree 
                program at a qualified institution of higher education, 
                as determined by the Director;
                    \item a student pursuing a degree on a less than 
                full-time basis, but not less than half-time basis; or
                    \item an AI faculty member on sabbatical to advance 
                knowledge in the field; and
\end{enumerate}
            \item accept the terms of a scholarship under this section.
\end{enumerate}
    \section{Conditions of Support.}\label{sec:support}
\begin{enumerate}
            \item {\bf In general.}---As a condition of receiving a scholarship 
        under this section, a recipient shall agree to provide the 
        qualified institution of higher education with annual 
        verifiable documentation of post-award employment and up-to-date contact information.
            \item {\bf Terms.}\label{enum:terms}---A scholarship recipient under this section 
        shall be liable to the United States as provided in \hyperref[sec:amount]{subsection~(k)} if the individual---
\begin{enumerate}
                    \item fails to maintain an acceptable level of 
                academic standing at the applicable institution of 
                higher education, as determined by the Director;
                    \item is dismissed from the applicable institution of 
                higher education for disciplinary reasons;
                    \item withdraws from the eligible degree program 
                before completing the program;
                    \item \label{enum:postaward}declares that the individual does not intend to 
                fulfill the post-award employment obligation under this 
                section; or
                    \item fails to fulfill the post-award employment 
                obligation of the individual under this section.
\end{enumerate}
\end{enumerate}
    \section{Monitoring Compliance.}As a condition of participating in the 
program, a qualified institution of higher education shall---
\begin{enumerate}
            \item enter into an agreement with the Director to monitor 
        the compliance of scholarship recipients with respect to their 
        post-award employment obligations; and
            \item provide to the Director, on an annual basis, the post-award employment documentation required under \hyperref[sec:support]{subsection~(i)} 
        for scholarship recipients through the completion of their 
        post-award employment obligations.
\end{enumerate}
    \section{Amount of Repayment.}\label{sec:amount}
\begin{enumerate}
            \item {\bf Less than 1 year of service.}---If a circumstance 
        described in \hyperref[enum:terms]{subsection~(i)(2)} occurs before the completion of 
        1 year of a post-award employment obligation under this 
        section, the total amount of scholarship awards received by the 
        individual under this section shall---
\begin{enumerate}
                    \item be repaid; or
                    \item be treated as a loan to be repaid in accordance \\
                with \hyperref[sec:repay]{subsection~(l)}.
\end{enumerate}
            \item 1 or more years of service.---If a circumstance 
        described in \hyperref[enum:postaward]{subparagraph (D) or (E) of subsection~(i)(2)} 
        occurs after the completion of 1 or more years of a post-award 
        employment obligation under this section, the total amount of 
        scholarship awards received by the individual under this 
        section, reduced by the ratio of the number of years of service 
        completed divided by the number of years of service required, 
        shall---
\begin{enumerate}
                    \item be repaid; or
                    \item be treated as a loan to be repaid in accordance \\
                with \hyperref[sec:repay]{subsection~(l)}.
\end{enumerate}
\end{enumerate}
    \section{Repayments.}\label{sec:repay}A loan described in \hyperref[sec:amount]{subsection~(k)} shall---
\begin{enumerate}
            \item be treated as a Federal Direct Unsubsidized Stafford 
        Loan\footnote{These are fixed interest rate loans at 4.99\% for undergraduate borrowers and 6.54\% for graduate or professional borrowers at the time of this writing.} under part D of title IV of the Higher Education Act of 
        1965 (20 U.S.C. 1087a et seq.); and
            \item be subject to repayment, together with interest thereon 
        accruing from the date of the scholarship award, in accordance 
        with terms and conditions specified by the Director (in 
        consultation with the Secretary of Education).
\end{enumerate}
    \section{Collection of Repayment.}
\begin{enumerate}
            \item {\bf In general.}---In the event that a scholarship recipient 
        is required to repay the scholarship award under this section, 
        the qualified institution of higher education providing the 
        scholarship shall---
\begin{enumerate}
                    \item determine the repayment amounts and notify the 
                recipient and the Director of the amounts owed; and
                    \item collect the repayment amounts within a period 
                of time as determined by the Director, or the repayment 
                amounts shall be treated as a loan in accordance with 
                \hyperref[sec:repay]{subsection~(l)}.
\end{enumerate}
            \item {\bf Returned to treasury.}---Except as provided in paragraph~(3), any repayment under this subsection shall be returned to 
        the Treasury of the United States.
            \item {\bf Retain percentage.}---A qualified institution of higher 
        education may retain a percentage of any repayment the 
        institution collects under this subsection to defray 
        administrative costs associated with the collection. The 
        Director shall establish a fixed percentage that will apply to 
        all eligible entities, and may update this percentage as 
        needed, in the determination of the Director.
\end{enumerate}
    \section{Exceptions.}The Director may provide for the partial or total 
waiver or suspension of any service or payment obligation by an 
individual under this section whenever compliance by the individual 
with the obligation is impossible or would involve extreme hardship to 
the individual, or if enforcement of such obligation with respect to 
the individual would be unconscionable.
    \section{Public Information.}
\begin{enumerate}
            \item {\bf Evaluation.}---The Director, in coordination with the 
        Director of the Office of Personnel Management, shall annually 
        evaluate and make public, in a manner that protects the 
        personally identifiable information of scholarship recipients, 
        information on the success of recruiting individuals for 
        scholarships under this section and on hiring and retaining 
        those individuals in the public sector AI workforce, including 
        information on---
\begin{enumerate}
                    \item placement rates;
                    \item where students are placed, including job titles 
                and descriptions;
                    \item salary ranges for students not released from 
                obligations under this section;
                    \item how long after graduation students are placed;
                    \item how long students stay in the positions they 
                enter upon graduation;
                    \item how many students are released from 
                obligations; and
                    \item what, if any, remedial training is required.
\end{enumerate}
            \item {\bf Reports.}---The Director, in coordination with the Office 
        of Personnel Management, shall submit, not less frequently than 
        once every 3 years, to the Committee on Homeland Security and 
        Governmental Affairs of the Senate, the Committee on Commerce, 
        Science, and Transportation of the Senate, the Committee on 
        Science, Space, and Technology of the House of Representatives, 
        and the Committee on Oversight and Reform of the House of 
        Representatives a report, including the results of the 
        evaluation under paragraph~(1) and any recent statistics 
        regarding the size, composition, and educational requirements 
        of the Federal AI workforce.
            \item {\bf Resources.}---The Director, in coordination with the 
        Director of the Office of Personnel Management, shall provide 
        consolidated and user-friendly online resources for prospective 
        scholarship recipients, including, to the extent practicable---
\begin{enumerate}
                    \item searchable, up-to-date, and accurate 
                information about participating institutions of higher 
                education and job opportunities related to the AI 
                field; and
                    \item a modernized description of AI careers.
\end{enumerate}
\end{enumerate}
    \section{Refresh.}Not less than once every 2 years, the Director, in 
coordination with the Director of the Office of Personnel Management, 
shall review and update the Federal AI Scholarship-for-Service Program 
to reflect advances in technology.

\appendix

\renewcommand{\thesection}{\Roman{section}}

    \section{Artificial Intelligence Defined.}The term \label{appendix:ai}
``artificial intelligence'' includes the following:
\begin{enumerate}
            \item Any artificial system that performs tasks under varying 
        and unpredictable circumstances without significant human 
        oversight, or that can learn from experience and improve 
        performance when exposed to data sets.
            \item An artificial system developed in computer software, 
        physical hardware, or other context that solves tasks requiring
        human-like perception, cognition, planning, learning, 
        communication, or physical action.
            \item An artificial system designed to think or act like a 
        human, including cognitive architectures and neural networks.
            \item A set of techniques, including machine learning, that is 
        designed to approximate a cognitive task.
            \item An artificial system designed to act rationally, 
        including an intelligent software agent or embodied robot that 
        achieves goals using perception, planning, reasoning, learning, 
        communicating, decision making, and acting.
\end{enumerate}

\end{document}